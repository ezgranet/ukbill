\documentclass{article}
\usepackage{verbatim}
\usepackage{xcolor}
\usepackage{luaquotes}
\usepackage{longtable}
\definecolor{darkspringgreen}{rgb}{0.09, 0.45, 0.27}
\definecolor{dsg}{rgb}{0.09, 0.45, 0.27}
\usepackage[hidelinks]{hyperref}
\usepackage{hologo}
\usepackage[british]{babel}
\usepackage[useregional]{datetime2}
\DTMlangsetup[en-GB]{ord=omit}
\definecolor{LightGray}{gray}{0.9}
%\usepackage{mathpazo}
\usepackage{fontspec}
\IfFontExistsTF{Palatine Parliamentary}{\setromanfont[RawFeature={+onum,+pnum},%
BoldFont={Palatine Parliamentary Bold},
ItalicFont={Palatine Parliamentary Italic}]{Palatine Parliamentary Regular}
}{\setromanfont[RawFeature={+onum,+pnum}]{TeX Gyre PagellaX}}
\setmonofont[Scale=.9]{Source Code Pro}
%\newfontface\primeback[Scale=1.01]{Libertinus Serif}
\newcommand{\primeback}{}

\newfontface\boxy{DejaVu Sans}
\newcommand{\thebox}{{\boxy ▯}}
%\usepackage[firstnumber=last]{fancyvrb}
\usepackage{minted}
\date{\today\\\smallskip\ttfamily Version 1.0.1}
\author{Elijah Z Granet\thanks{e-mail: \href{mailto:ezg21@cantab.ac.uk}{\ttfamily ezg21@cantab.ac.uk}}}

\title{\texttt{ukbill}:\\A package for typesetting UK legislation}

\begin{document}
\maketitle
\tableofcontents
\clearpage
\section{Overview}
This is a class for typesetting bills in the standard used by the Parliament of the United Kingdom. It is meant to be of use to students of law and politics, to Parliamentary agents and lawyers drafting private bills, and to aid in the construct of proposals for law reform. The use of this package requires the libre font `Palatine Parliamentary', which imitates the official typeface of the UK Parliament.\footnote{The font is available at this link: \url{https://github.com/ezgranet/palatine-parliamentary}}
\section{Usage}
For an example of the usage of the package, please see the attached `\texttt{immigration-bill.tex}' example in the package archive.

\subsection{Calling the package}
Call the package with {\color{darkspringgreen}\ttfamily \verb|\documentclass{ukbill}|}
\subsection{Declaring variables}
In your preamble, declare  the following variables to populate your bill. 
\begin{center}
\begin{tabular}{lp{2.5in}}
\bfseries Variable name & \bfseries Explanation\\
\ttfamily \verb|\drafter{}| & The drafter of the bill\\
\ttfamily \verb|\billcopyright{}| & Copyright notice at the end of the bill \\
\ttfamily \verb|\publishedby{}| & The publisher of the bill for notice at the end of the bill \\
\ttfamily\verb|\billtitle{}| & The title of the bill\\
\ttfamily \verb|billto{}| & { The purpose or long title of the bill}\\
\ttfamily \verb|\humanrights{}| & The statutorily required statement under the Human Rights Act 1998\\
\ttfamily \verb|\exptitle{}| & The title of the explanatory section—if you have extra notes added separately, title it `Explanatory Notes'; otherwise use `Explanatory Memorandum'\\
\ttfamily \verb|\exptext{}| & The content of the explanatory memorandum (or notice about separate explanatory notes)\\
\ttfamily \verb|\billnum{}| & The number of the bill for the cover sheet and back page\\
\ttfamily \verb|\whereas{}| &
This variable is needed \textbf{only} in the \texttt{private} option of the class. It contains  the recitals, best formatted as a numbered list 
\end{tabular}

\end{center}


\subsection{Bill parts}
The command {\color{darkspringgreen}\ttfamily\verb|\chapter|} divides the  main content of the bill up into appropriate parts noted in the Table of Contents. For example, `Introductory Provisions'.  
\subsection{Sections}
The section is the main unit of legislation and is titled with {\color{darkspringgreen}\ttfamily\verb|\section|}.  Most sections will be numbered and numbered subsections use the {\color{darkspringgreen}\ttfamily\verb|numstat|} environment as follows:
\begin{minted}[
frame=lines,
framesep=2mm,
baselinestretch=1.2,
bgcolor=LightGray,
fontsize=\footnotesize,
breaklines,
firstnumber=last
]
{latex}
\begin{numstat}
	\item The first provision
	\item The second provision
\end{numstat}
\end{minted}

Often, these sections will be nested with alphabetical subsubsections, Roman paragraphs, and double alphabetical (\textit{ie}, `aa') subparagraphs. These are called with  {\color{darkspringgreen}\ttfamily\verb|alphstat|}, {\color{darkspringgreen}\ttfamily\verb|romstat|}, and {\color{darkspringgreen}\ttfamily\verb|twoalphstat|}, respectively.
\begin{minted}[
frame=lines,
framesep=2mm,
baselinestretch=1.2,
bgcolor=LightGray,
fontsize=\footnotesize,
breaklines,
firstnumber=last
]
{latex}
\begin{numstat}
	\item A subsection
	\begin{alphstat}
		\item Nesting
		\begin{romstat}
			\item Even more nesting
			\begin{twoalphstat}
				\item this is too much nesting now, lads	
			\end{twoalphstat}
  		\end{romstat}
	\end{alphstat}
	\item a subsection again
\end{numstat}
\end{minted}


Occasionally, a section will have only one provision, and therefore no numbering is needed. In this case, use the {\color{darkspringgreen}\ttfamily\verb|nostat|} environment.

\begin{minted}[
frame=lines,
framesep=2mm,
baselinestretch=1.2,
bgcolor=LightGray,
fontsize=\footnotesize,
breaklines,
firstnumber=last
]
{latex}
\begin{nostat}
	\item The first provision
	\item The second provision
\end{nostat}
\end{minted}

\subsection{Schedules}

Some legislation requires schedules appended to the body of main legislation. To begin typesetting schedules (as opposed to the preceding main content of the bill), use the command {\color{darkspringgreen}\ttfamily\verb|\startschedule|}

A schedule is then named by the command {\color{darkspringgreen}\ttfamily\verb|\schedule{Name}|}, while a Part (\textit{ie}, a subset of a schedule) is called using {\color{darkspringgreen}\ttfamily\verb|\schdpart{name}|}

\subsection{The `private' Option}

Private bills have a different enacting formula and also make use of recitals. Drafting private bills therefore requires you to call the class with the option {\color{darkspringgreen}\ttfamily\verb|\documentclass[private]{ukbill}|}
	
	\section{Future Development }
	The package's online repository is the best place to report bugs, feature requests, or other contributions, and is located at: \\\url{github.com/ezgranet/ukbill}.  
	
	\section{Licence}
	This project is licensed under the Latex Public Project Licence version 1.3\textit{c}. This documentation is copyright of the author but licensed under CC-BY-SA 4.0. 
	\section{Version History}
	\subsection{\normalfont\texttt{1.0.1}}
\ttfamily 12 December 2022: Fixes to example documentation

\subsection{\normalfont\texttt{1.0.0}}
\ttfamily 1 December 2022: Class Creation


	\clearpage\section{Implementation}
	
\begin{minted}[
frame=lines,
framesep=2mm,
baselinestretch=1.2,
bgcolor=LightGray,
fontsize=\footnotesize,
linenos,
breaklines,
firstnumber=last
]
{latex}
\NeedsTeXFormat{LaTeX2e}
\def\ukbillversionnumber{1.0.0}
\ProvidesClass{ukbill}[2022-12-01 A Class for Legal Notes]
  % !TeX program = lualatex                                   
% !TeX encoding = utf8
% This work may be distributed and/or modified under the 
% conditions of the LaTeX Project Public License, either version 1.3c
% of this license or (at your option) any later version.
% The latest version of this license is in
%   http://www.latex-project.org/lppl.txt
% and version 1.3c or later is part of all distributions of LaTeX 
% version 2005/12/01 or later.
%
% This work has the LPPL maintenance status `maintained'.
%
% The Current Maintainer of this work is Elijah Z Granet


\LoadClass[a4paper,12pt]{memoir}
%%%%%%%%%%%%%%%%%%%%%%%%%%%
%%%%%%%%%%%%%%%%%%%%%%%%%%%
% schedule divisions
%%%%%%%%%%%%%%%%%%%%%%%%%%%
\newcommand{\schdpart}[1]{\subsubsection{#1}}
\newcommand{\startschedule}{\clearpage
\setcounter{schedon}{1}
%\setcounter{subsection}{}
\setcounter{part}{0}
\begin{center}
	\large\textsc{SCHEDULES}
\end{center}
\addcontentsline{toc}{chapter}{\rule{.45\textwidth}{1pt}}
}
%%%%%%%%%%%%%%%%%%%%%%%%%%%
%%%%%%%%%%%%%%%%%%%%%%%%%%%
%%%%%%%%%%%%%%%%%%%%%%%%%%%
% Variables
%%%%%%%%%%%%%%%%%%%%%%%%%%%
%%%%%%%%%%%%%%%%%%%%%%%%%%%
\makeatletter
\newcommand{\drafter}[1]{\def\@drafter{#1}}
\newcommand{\printdrafter}{\@drafter}
\def\@drafter{\@latex@warning@no@line{No \noexpand\drafter given}}
\newcommand{\billcopyright}[1]{\def\@billcopyright{#1}}
\newcommand{\printbillcopyright}{\@billcopyright}
\def\@billcopyright{\@latex@warning@no@line{No \noexpand\billcopyright given}}
\newcommand{\billto}[1]{\def\@billto{#1}}
\newcommand{\printbillto}{\@billto}
\def\@billto{\@latex@warning@no@line{No \noexpand\billto given}}
\newcommand{\whereas}[1]{\def\@whereas{#1}}
\newcommand{\printwhereas}{\@whereas}

\newcommand{\publishedby}[1]{\def\@publishedby{#1}}
\newcommand{\printpublishedby}{\@publishedby}
\def\@publishedby{\@latex@warning@no@line{No \noexpand\publishedby given}}
\newcommand{\billtitle}[1]{\def\@billtitle{#1}}
\newcommand{\printbilltitle}{\@billtitle}
\def\@billtitle{\@latex@warning@no@line{No \noexpand\billtitle given}}
\newcommand{\humanrights}[1]{\def\@humanrights{#1}}
\newcommand{\printhumanrights}{\@humanrights}
\def\@humanrights{\@latex@warning@no@line{No \noexpand\humanrightsgiven}
}
\newcommand{\exptitle}[1]{\def\@exptitle{#1}}
\newcommand{\printexptitle}{\@exptitle}
\def\@exptitle{\@latex@warning@no@line{No \noexpand\exptitle given}}
\newcommand{\exptext}[1]{\def\@exptext{#1}}
\newcommand{\printexptext}{\@exptext}
\def\@exptext{\@latex@warning@no@line{No \noexpand\exptext given}}
\newcommand{\billnum}[1]{\def\@billnum{#1}}
\newcommand{\printbillnum}{\@billnum}
\def\@billnum{\@latex@warning@no@line{No \noexpand\billnum given}}


\makeatother
%%%%%%%%%%%%%%%%%%%%%%%%%%%
%%%%%%%%%%%%%%%%%%%%%%%%%%%
% hyphenation
%%%%%%%%%%%%%%%%%%%%%%%%%%%
%%%%%%%%%%%%%%%%%%%%%%%%%%%
\RequirePackage[none]{hyphenat}
%%%%%%%%%%%%%%%%%%%%%%%%%%%
%%%%%%%%%%%%%%%%%%%%%%%%%%%
% list organisation
%%%%%%%%%%%%%%%%%%%%%%%%%%%
%%%%%%%%%%%%%%%%%%%%%%%%%%%
\RequirePackage{paralist}
  \let\itemize\compactitem
  \let\enditemize\endcompactitem
  \let\enumerate\compactenum
  \let\endenumerate\endcompactenum
  \let\description\compactdesc
  \let\enddescription\endcompactdesc
  \pltopsep=-12pt
  \plitemsep=0pt
  \plparsep=0pt
\newcounter{schedcount}
\counterwithout{section}{chapter}

\counterwithin{schedcount}{subsubsection}
\setsecnumdepth{subsubsection}
\def\tocmark{\markboth{\MakeTextUppercase{}}{}}
\RequirePackage[absolute]{textpos}
\setlength{\TPHorizModule}{10mm}
\setlength{\TPVertModule}{10mm}
\TPGrid[15mm,1mm]{8}{9}                        % Divide page in 9x9 grid 
\DeclareRobustCommand{\Leftblock}{{
\begin{textblock}{10}(0.5,1.8)\footnotesize\itshape 5\end{textblock}
\begin{textblock}{10}(0.5,2.7)\footnotesize\itshape 10\end{textblock}
\begin{textblock}{10}(0.5,3.6)\footnotesize\itshape 15\end{textblock}
\begin{textblock}{10}(0.5,4.5)\footnotesize\itshape 20\end{textblock}
\begin{textblock}{10}(0.5,5.4)\footnotesize\itshape 25\end{textblock}
\begin{textblock}{10}(0.5,6.3)\footnotesize\itshape 30\end{textblock}
\begin{textblock}{10}(0.5,7.2)\footnotesize\itshape 35\end{textblock}
}}
\DeclareRobustCommand{\Rightblock}{{
\begin{textblock}{10}(7.75,1.8)\footnotesize\itshape 5\end{textblock}
\begin{textblock}{10}(7.75,2.7)\footnotesize\itshape 10\end{textblock}
\begin{textblock}{10}(7.75,3.6)\footnotesize\itshape 15\end{textblock}
\begin{textblock}{10}(7.75,4.5)\footnotesize\itshape 20\end{textblock}
\begin{textblock}{10}(7.75,5.4)\footnotesize\itshape 25\end{textblock}
\begin{textblock}{10}(7.75,6.3)\footnotesize\itshape 30\end{textblock}
\begin{textblock}{10}(7.75,7.2)\footnotesize\itshape 35\end{textblock}
}}
\DeclareRobustCommand{\Firstblock}{{
%\begin{textblock}{10}(7.75,1.8)\footnotesize\itshape 5\end{textblock}
%\begin{textblock}{10}(7.75,2.7)\footnotesize\itshape 10\end{textblock}
%\begin{textblock}{10}(7.75,3.6)\footnotesize\itshape 15\end{textblock}
\begin{textblock}{10}(7.75,4.5)\footnotesize\itshape 5\end{textblock}
\begin{textblock}{5}(7.75,5.4)\footnotesize\itshape 10\end{textblock}
\begin{textblock}{10}(7.75,6.3)\footnotesize\itshape 15\end{textblock}
\begin{textblock}{10}(7.75,7.2)\footnotesize\itshape 20\end{textblock}
}}

\RequirePackage{ccicons}
 \makepagestyle{bill}
  \makepagestyle{billfirst}

 \makeatletter
 \makepsmarks {bill}{
\nouppercaseheads}
\makeatother
\makeatletter
 \renewcommand{\bookpagemark}[1]{\itshape\thetitle\\}
  \makepsmarks {billfirst}{
\nouppercaseheads}
\makeatother
 \renewcommand{\bookpagemark}[1]{\noindent\itshape\thetitle\\}

% \renewcommand{\partpagemark}[1]{\itshape\thetitle\\}
 \makepagestyle{front}

 \makepagestyle{sched}
 \makeatletter
 \makepsmarks {sched}{
\nouppercaseheads}
\makeatother
 \renewcommand{\bookpagemark}[1]{\itshape\thetitle\\}
% \renewcommand{\partpagemark}[1]{\itshape\thetitle\\}
\clearmark{chapter}
\clearmark{section}

\createmark{part}{both}{shownumber}{\partname\space}{.\space}

 
 \RequirePackage[margin=3.25cm,headheight=100pt]{geometry}
 \makeatletter
 \makeevenfoot{front}{}{}{\printbillnum}
  \makeoddfoot{front}{}{}{\printbillnum}
  \makeatother

\renewcommand{\bookname}{\printbilltitle}
\makeheadrule{bill}{\textwidth}{1pt}
 \makeevenhead{bill}{\Rightblock\normalsize\thepage\ifnum\value{part}>0\\\else\fi\ifnum\value{schedon}=1\\\else\fi\ifnum\value{subsubsection}>0\vskip 12pt\else\fi}{}{\footnotesize\itshape\bookname
   \ifnum\value{schedon}=1
\\ \itshape Schedule \arabic{subsection}\hspace{1ex}—\hspace{1ex}\currentsubsection \else\fi
  \ifnum\value{part}>0
\\Part \arabic{part}\hspace{1ex}—\hspace{1ex}\rightmark
 \else\fi\ifnum\value{subsubsection}>0\\Part \arabic{subsubsection}\hspace{1ex}—\hspace{1ex}\currentsubsubsection\else\fi}
  \makeoddhead{bill}{\Rightblock\footnotesize\itshape\bookname
   \ifnum\value{schedon}=1
\\ \itshape Schedule \arabic{subsection}\hspace{1ex}—\hspace{1ex}\currentsubsection \else\fi
  \ifnum\value{part}>0
\\Part \arabic{part}\hspace{1ex}—\hspace{1ex}\rightmark
 \else\fi\ifnum\value{subsubsection}>0\\Part \arabic{subsubsection}\hspace{1ex}—\hspace{1ex}\currentsubsubsection\else\fi}{}{\normalsize\thepage\ifnum\value{part}>0\\\else\fi\ifnum\value{schedon}=1\\\else\fi\ifnum\value{subsubsection}>0\vskip 12pt\else\fi}
 \makeoddfoot{bill}{}{}{}
%\renewcommand{\subsectionmark}[1]{#1}




%%%%%%%%%%%%%%%%%%%%%%%%%%%
%%%%%%%%%%%%%%%%%%%%%%%%%%%
% bill-first

%%%%%%%%%%%%%%%%%%%%%%%%%%%
%%%%%%%%%%%%%%%%%%%%%%%%%%%
\makeheadrule{billfirst}{\textwidth}{1pt}
 \makeevenhead{billfirst}{\Firstblock\normalsize\thepage\ifnum\value{part}>0\\\else\fi\ifnum\value{schedon}=1\\\else\fi\ifnum\value{subsubsection}>0\vskip 12pt\else\fi}{}{\footnotesize\itshape\bookname
   \ifnum\value{schedon}=1
\\ \itshape Schedule \arabic{subsection}\hspace{1ex}—\hspace{1ex}\currentsubsection \else\fi
  \ifnum\value{part}>0
\\Part \arabic{part}\hspace{1ex}—\hspace{1ex}\rightmark
 \else\fi\ifnum\value{subsubsection}>0\\Part \arabic{subsubsection}\hspace{1ex}—\hspace{1ex}\currentsubsubsection\else\fi}
  \makeoddhead{billfirst}{\Firstblock\footnotesize\itshape\bookname
   \ifnum\value{schedon}=1
\\ \itshape Schedule \arabic{subsection}\hspace{1ex}—\hspace{1ex}\currentsubsection \else\fi
  \ifnum\value{part}>0
\\Part \arabic{part}\hspace{1ex}—\hspace{1ex}\rightmark
 \else\fi\ifnum\value{subsubsection}>0\\Part \arabic{subsubsection}\hspace{1ex}—\hspace{1ex}\currentsubsubsection\else\fi}{}{\normalsize\thepage\ifnum\value{part}>0\\\else\fi\ifnum\value{schedon}=1\\\else\fi\ifnum\value{subsubsection}>0\vskip 12pt\else\fi}
 \makeoddfoot{billfirst}{}{}{}
%\renewcommand{\subsectionmark}[1]{#1}



\renewcommand{\chaptermark}[1]{#1}
\cftpagenumbersoff{part}
\cftpagenumbersoff{chapter}
\cftpagenumbersoff{subsection}
\cftpagenumbersoff{subsubsection}
\renewcommand{\cftsubsectionpresnum}{\flushleft Schedule\hspace{1ex}}% Prefix to number for \subsection in ToC

\renewcommand{\cftsubsubsectionpresnum}{Part\hspace{1ex}}% Prefix to number for \subsection in ToC

\renewcommand{\thesubsection}{\arabic{subsection}}
\renewcommand{\thesubsubsection}{\arabic{subsubsection}}
\renewcommand{\cftsubsectionaftersnum}{\hspace{1ex}—\hspace{1ex}}
\renewcommand{\cftsubsubsectionaftersnum}{\hspace{1ex}—\hspace{1ex}}

\settocdepth{subsubsection}
\cftpagenumbersoff{section}
\makeatletter
\let\stdl@chapter\l@chapter
\renewcommand*{\l@chapter}[2]{\stdl@chapter{\centerline{#1}}{#2}}
\makeatother
\cftsetindents{subsection}{1.75cm}{10em}
\cftsetindents{subsubsection}{1.75cm}{10em}

\renewcommand{\cftsubsectionpresnum}{Schedule\hfill}
\renewcommand{\cftsubsectionaftersnum}{\quad—\quad}
\renewcommand{\cftsubsubsectionpresnum}{\qquad Part\hfill}
\renewcommand{\cftsubsubsectionaftersnum}{\quad—\quad}

%\setlength\cftsubsectionnumwidth{7.3em}
%\setlength\cftsubsubsectionnumwidth{4em}

\renewcommand{\title}{\printbilltitle}
\RequirePackage{changepage}   % for the adjustwidth environment
\renewcommand{\part}[1]{\stepcounter{part}\markright{#1}
\begin{center}
\printpartname\hspace{1ex}\arabic{part}\\
\printparttitle{#1}\normalfont\normalsize\\\end{center}
\addcontentsline{toc}{part}{\normalsize\bfseries\printpartname\hspace{1ex}%
\arabic{part}\normalfont\\
\centerline{\normalsize\normalfont\scshape\printparttitle{#1}}}
}
  \renewcommand\contentsname{CONTENTS}%

\renewcommand{\chapter}[1]{\stepcounter{chapter}\markright{#1}
\begin{center}
\itshape\printchaptertitle{#1}\normalfont\normalsize\\%
\end{center}
\addcontentsline{toc}{chapter}{\normalsize\itshape
\centerline{\normalsize\normalfont\itshape\printchaptertitle{#1}}}
}

\renewcommand{\partnamefont}{\normalfont\centering\bfseries\scshape}
\renewcommand{\parttitlefont}{\normalfont\centering\scshape}
\makeatletter
\renewcommand{\partnumberline}[1]{\hfil\hspace\@tocrmarg #1~}
\makeatother

\renewcommand{\partnumfont}{\normalfont\bfseries\centering\normalsize\scshape}
\renewcommand{\partnamenum}{\normalfont\bfseries\centering\normalsize\scshape}
\RequirePackage{setspace}
\renewcommand{\chapterheadstart}{\vspace*{\beforechapskip}}
\renewcommand{\printchaptername}{\chapnamefont \@chapapp}
\renewcommand{\chapternamenum}{\space}
\renewcommand{\printchapternum}{\chapnumfont \thechapter}
\renewcommand{\afterchapternum}{\par\nobreak\vskip \midchapskip}
\renewcommand{\printchapternonum}{}
\renewcommand{\chapterheadstart}{}
\renewcommand{\printchaptername}{}
\renewcommand{\chapternamenum}{}
\renewcommand{\printchapternum}{}
\renewcommand{\afterchapternum}{}
\setlength{\parskip}{0pt}
\renewcommand{\printchaptertitle}[1]{\chaptitlefont #1}
\renewcommand{\afterchaptertitle}{\par\nobreak\vskip \afterchapskip}
\renewcommand{\printtoctitle}[1]{\large\scshape\centerline{\MakeUppercase{#1}}}
\renewcommand{\chapnamefont}{\normalfont\centering\itshape}
\renewcommand{\chapnumfont}{\normalfont\centering\itshape}
\renewcommand{\chaptitlefont}{\normalfont\centering\itshape}

\setlength{\beforechapskip}{12pt}
\setlength{\midchapskip}{0pt}
\setlength{\afterchapskip}{12pt}
\setbeforesecskip{-1em}
\setaftersecskip{-1em}
\setbeforesubsecskip{-1em}
\setaftersubsecskip{-1em}
\renewcommand*{\thesection}{\arabic{section}}
\setsecnumformat{\csname the#1\endcsname\hspace{5ex}}
%\setsubsecnumformat{\csname the#1\endcsname\hspace{5ex}}
%\setsubsecheadstyle{\bfseries}
%\setsubsecheadstyle{\scshape\centering}
%\setsubsectionnumformat{SCHEDULE\quad\csname the#1\endcsname\newline}
\RequirePackage[compact]{titlesec}
\titleformat{\subsection}[display]
  {\normalfont\centering\scshape}{SCHEDULE \hspace{2ex }\thesubsection}{5pt}{}
\titleformat{\subsubsection}[display]
  {\normalfont\centering\scshape}{Part \thesubsubsection}{5pt}{}

\setsecheadstyle{\centering\bfseries}
\renewcommand{\part}[1]{\stepcounter{part}\markright{#1}
\begin{center}
\normalfont\normalsize\itshape\printchaptertitle{#1}\normalfont\normalsidze\\\end{center}
\addcontentsline{toc}{chapter}{
\centerline{\normalsize\normalfont\itshape\printchaptertitle{#1}}}
}
\RequirePackage{fontspec}
\IfFontExistsTF{Palatine Parliamentary}{
\setmainfont[
Scale=.98, 
SmallCapsFeatures={LetterSpace=10,RawFeature={+smcp,},},
BoldFeatures = {SmallCapsFont= {PalatineP-Bold},SmallCapsFeatures={%
RawFeature={+smcp,}%
}},
BoldFont={PalatineP-Bold}, 
 ItalicFont={PalatineP-Italic},
 BoldItalicFont={PalatineP-BoldItalic}]{PalatineP-Regular}
\newfontface{\extfont}[SmallCapsFont={Times New Roman}]{Times New Roman}}{\ClassWarning{Please install the Palatine Parliamentary Font}}

\RequirePackage[english]{babel}
%\RequirePackage{csquotes}
%\MakeOuterQuote{"}
%%\DeclareQuoteStyle{english}%
%%    {{\extfont\textquotedblleft}}
%%    [\extfont\textquotedblleft]
%%    {{\extfont\textquotedblright}}
%%        [0.05em]
%%    {{\extfont\textquoteleft}}
%%    [{\extfont\textquoteleft}]
%%    {{\extfont\textquoteright}}
\RequirePackage{enumitem}
\newenvironment{statquote}%
{
\begin{list}{}%
{%
\setlength{\topsep}{0ex}%
\setlength{\partopsep}{0ex}%
\setlength{\parsep}{0.5ex}%
\setlength{\itemsep}{i}%
\addtolength{\leftmargin}{3em}%
\addtolength{\rightmargin}{3em}%
}%
\item[]}%
{\end{list}}
\newenvironment{nostat}{\begin{enumerate}[nosep,leftmargin=1.25cm,%
labelindent=5pt,itemindent=-15pt,label=]}{\end{enumerate}}
\newenvironment{numstat}%
{%
%\vspace{-9ex}%
\begin{enumerate}[nosep,partopsep=0pt,
%parsep=0.5ex,
%itemsep=1ex,
labelsep=3ex,
leftmargin=4em,rightmargin=1.5em,label=(\arabic*)]%
\item[]}%
{\end{enumerate}%
\vspace{1ex}\normalfont}%
\newenvironment{schumstat}%
{%
%\vspace{-9ex}%
\begin{enumerate}[nosep,partopsep=0pt,
%parsep=0.5ex,
itemsep=0ex,
labelsep=0ex,
leftmargin=0em,rightmargin=0em,label=]%
\item[]}%
{\end{enumerate}%
\vspace{1ex}\normalfont}%

\newcounter{alphcount}

\newenvironment{instatquote}%
{%
\setcounter{alphcount}{0}
\vspace{-3ex}%
\begin{list}{}%
{%
\setlength{\topsep}{0ex}%
\setlength{\partopsep}{1ex}%
\setlength{\parsep}{0.5ex}%
\setlength{\itemsep}{0ex}%
}%
\item[]}%
{\end{list}%
}%
\newenvironment{alphstat}%
{%
\vspace{-3ex}%
\begin{enumerate}[leftmargin=7.5ex,topsep=0ex,partopsep=1ex,parsep=0.5ex,%
itemsep=0ex,labelsep=3ex,label=({\alph*})]%

\item[]}%
{\end{enumerate}%
}%
\newenvironment{romstat}%
{%
\vspace{-3ex}%
\begin{enumerate}[topsep=0ex,partopsep=1ex,parsep=0.5ex,itemsep=0ex,%
label=(\roman*)]%

\item[]}%
{\end{enumerate}%
}%
\newenvironment{twoalphstat}%
{%
\vspace{-3ex}%
\begin{enumerate}[leftmargin=7.5ex,topsep=0ex,partopsep=1ex,parsep=0.5ex,%
itemsep=0ex,labelsep=3ex,label=({\alph*}{\alph*})]%

\item[]}%
{\end{enumerate}%
}%

\newcommand*\statquotelabel[1]{}

\newcommand{\stat}[2]{
\item[(#1)]#2
}

\newcommand{\stathead}[2]{
\textbf{#1}\hspace{5ex}\textbf{#2}
}




\setlength{\parindent}{0pt}
\RequirePackage{lettrine}
\setcounter{DefaultLines}{2}
%\renewcommand{\DefaultLraise}{0.05}
%\renewcommand{\DefaultLoversize}{.07}
%\renewcommand{\LettrineFontHook}{\initials}
%\RequirePackage{initials}
%\RequirePackage{Romantik}%\romantik{} 
%\RequirePackage{Royal}%\royal

\newcommand{\intl}[2]{\lettrine{#1}{\hspace{1ex}\textsc{#2}}}
\renewcommand{\maketitle}{\thispagestyle{front}
  \begin{center}
  {\Huge\textbf{\title}}\\\rule{\textwidth}{1pt}\\\rule[16pt]{\textwidth}{2pt}
  \end{center}
  }
  \newcommand{\schedule}[1]{\subsection{#1}}
\newcounter{schedon}
\setcounter{schedon}{0}

\newcommand{\currentsubsection}{}
\let\oldsubsection\subsection
\renewcommand{\subsection}[1]{\oldsubsection{#1}\renewcommand{%
\currentsubsection}{#1}}

\newcommand{\currentsubsubsection}{}
\let\oldsubsubsection\subsubsection
\renewcommand{\subsubsection}[1]{\oldsubsubsection{#1}\renewcommand{%
\currentsubsubsection}{#1}}
%%%%%%%%%%%%%%%%%%%%%%%%%%%
%%%%%%%%%%%%%%%%%%%%%%%%%%%
% enacting
%%%%%%%%%%%%%%%%%%%%%%%%%%%
%%%%%%%%%%%%%%%%%%%%%%%%%%%
\newcommand{\enactingformula}{
\intl{B}{e it enacted} by the King's most Excellent Majesty, by and with the advice and consent of the Lords Spiritual and Temporal, and Commons, in this present Parliament assembled, and by the authority of the same, as follows:---}

%%%%%%%%%%%%%%%%%%%%%%%%%%%
%%%%%%%%%%%%%%%%%%%%%%%%%%%
%enacting option
%%%%%%%%%%%%%%%%%%%%%%%%%%%
%%%%%%%%%%%%%%%%%%%%%%%%%%%
\DeclareOption{private}{
\makeatletter
\renewcommand{\enactingformula}{
\intl{W}{hereas}---

\hspace{2em}\begin{minipage}{.75\textwidth}

\printwhereas
\end{minipage}

\medskip

\intl{M}{ay} it therefore please Your Majesty that it may be enacted, and be it enacted, by
the King’s Most Excellent Majesty, by and with the advice and consent of the Lords
Spiritual and Temporal, and Commons, in this present Parliament assembled, and by
the authority of the same, as follows:―
}


\makeatother

}

\ProcessOptions\relax


%%%%%%%%%%%%%%%%%%%%%%%%%%%
%%%%%%%%%%%%%%%%%%%%%%%%%%%
% begindoc
%%%%%%%%%%%%%%%%%%%%%%%%%%%
%%%%%%%%%%%%%%%%%%%%%%%%%%%
\AtBeginDocument{\emergencystretch 3em
\sloppy
\OnehalfSpacing
\pagestyle{front}
\pagenumbering{roman}
\maketitle
\medskip
\extfont
\begin{center}\scshape\MakeUppercase{\printexptitle}\end{center}

\printexptext
\medskip

\begin{center}\scshape EUROPEAN CONVENTION ON HUMAN RIGHTS\end{center}

\printhumanrights
\normalfont
\clearpage\maketitle
\begin{KeepFromToc}
\normalfont\normalsize
\tableofcontents
\end{KeepFromToc}

\vskip 3ex
\mainmatter
\pagenumbering{arabic}
\pagestyle{bill}\thispagestyle{billfirst}
\begin{center}
	\scshape a
	\medskip
	
	\normalfont\Huge\bfseries B\hspace{.5ex}I\hspace{.5ex}L\hspace{.5ex}L
	
	\medskip
	
	\normalfont\normalsize\scshape to\normalfont
	
\end{center}

	\medskip

\noindent\footnotesize\extfont\printbillto\normalsize \normalfont

\medskip 

\enactingformula

\medskip



}


\AtEndDocument{\cleardoublepage


    \setcounter{page}{0}
    \pagenumbering{arabic}
    \setcounter{page}{1}
\thispagestyle{front}

  \begin{center}
  {\Huge\textbf{\title}}\\\rule{\textwidth}{1pt}\\\rule[16pt]{\textwidth}{2pt}
  \end{center}

\begin{center}
	\normalfont\scshape a
	\medskip
	
	\normalfont\Huge\bfseries B\hspace{.5ex}I\hspace{.5ex}L\hspace{.5ex}L
	
	\vskip 3ex
	\end{center}
	
\normalsize To make provision in connection with the citizens of certain Commonwealth Realms
	
	\vskip 3ex
	\begin{center}
	
\normalsize\normalfont\itshape	Presented by \printdrafter

\vskip 3ex

\rule{.45\textwidth}{1pt}
\vspace{1ex}

\begin{minipage}{.4\textwidth}\centering
\itshape Ordered, by\normalfont\ \printdrafter\itshape\ to be Printed, \normalfont 19\itshape th March \normalfont 2020\itshape.
\end{minipage}

\vspace{1ex} 

\rule{.45\textwidth}{1pt}

\vskip 5ex

\normalfont\footnotesize\printbillcopyright


\vskip 1ex 

\normalfont\scshape\MakeUppercase{Published By \printpublishedby}
\end{center}
}
\end{minted}


\end{document}